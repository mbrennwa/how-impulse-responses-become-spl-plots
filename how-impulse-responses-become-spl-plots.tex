\documentclass[12pt,a4paper]{article}
\usepackage[margin=2.5cm]{geometry}
\usepackage{helvet}
\renewcommand{\familydefault}{\sfdefault}
\usepackage[helvet]{sfmath}
\usepackage{amsmath,amssymb}
\usepackage{enumitem}
\usepackage{titlesec}
\usepackage{graphicx}
\usepackage{hyperref}
\usepackage[parfill]{parskip}

\title{How Loudspeaker Impulse Responses Become Sound~Pressure~Level~(SPL) Plots:\\[0.7em]\textit{The Good, the Bad, and the Ugly}}
\author{Matthias S.~Brennwald}
\date{Document version \today}

\begin{document}
\maketitle

\section{Introduction}
\section*{Introduction}

Every loudspeaker designer or tinkerer wants to know how well their loudspeaker performs acoustically. Does it reproduce all frequencies evenly, or does it emphasise some and hide others? The answer lies in its SPL frequency response curve, which shows how loud the speaker is at each frequency.

Understanding and visualising the SPL response is central to almost every aspect of loudspeaker design and evaluation—from developing new drivers to characterising enclosures and interpreting room interactions. The goal seems simple: determine how a loudspeaker responds to different frequencies. The challenge lies in doing so accurately and efficiently, without (unintended) cheating.

An intuitive approach to determine the SPL curve of a loudspeaker would be to play one sine wave with a given frequency at a time, measure the resulting sound pressure level (SPL), repeat the process at the next frequency, and plot the SPL results across frequency. However, this method is very time-consuming and works reliably only in an anechoic environment, as room reflections and echoes can severely distort the measurement. Modern techniques achieve the same result in a single measurement step: they excite the loudspeaker with a short, broadband signal and use the Fourier transform to reveal the frequency response from the measured time-domain data.

This document introduces the concepts that link time-domain measurements (such as impulse responses) to frequency-domain data (such as SPL curves). It aims to provide a practical, intuitive explanation of what the Fourier transform does for loudspeaker measurements—and why understanding its limitations is just as important as applying it. Later sections explore both the strengths and the pitfalls of this process: the good, the bad, and occasionally the ugly.


\section{Time domain vs.\ frequency domain: the Fourier transform}

Assume a sound pressure signal evolving as a function of time, $p(t)$. The signal is recorded by a microphone and digitized by an A/D converter with sampling rate \(F_s\). As a result, we obtain $N$ discrete samples $p_i = p(t_i)$ at times
\[
t_i = (i-1)\tau, \qquad \tau = \frac{1}{F_s}, \quad i = 1, 2, \ldots, N.
\]

The idea of the Fourier transform is to describe each time-domain data point \(p_i\)
as a combination of sine waves, each with amplitude \(A_j\) and phase \(\varphi_j\),
at discrete frequencies
\[
f_j = \frac{F_s}{N}j, \quad j = 1, 2, \ldots, M,
\]
where \(M = (N - 1)/2\) if \(N\) is odd, or \(M = N/2 - 1\) if \(N\) is even.

The discrete Fourier expansion is then written as:
\begin{align}
\text{For odd } N:\quad
p_i &= A_0 + \sum_{j=1}^{M} A_j \sin\!\big(2\pi f_j t_i + \varphi_j\big), \\[6pt]
\text{For even } N:\quad
p_i &= A_0 + \sum_{j=1}^{M} A_j \sin\!\big(2\pi f_j t_i + \varphi_j\big)
      + A_{N/2}\cos\!\big(\pi\, t_i / \tau\big).
\end{align}

Each amplitude–phase pair \((A_j, \varphi_j)\) represents one Fourier component at frequency \(f_j = jF_s/N\). The $A_0$ coefficient is the DC component of the $p_i$ data. Note that the total number of coefficients $A_j$ (including $A_0$) and $\varphi_j$ together equals \(N\). In other words, the number of data points in the frequency domain is the same as in the time domain. The above equations therefore represent a fully determined system. The coefficients \((A_j, \varphi_j)\) can be uniquely obtained from the measured samples \(p_i\) by solving this system, typically using efficient numerical algorithms such as the
\emph{Fast Fourier Transform (FFT)}.

The Fourier transform is reversible. Given the frequency-domain data \((A_j, \varphi_j)\),
the corresponding time-domain signal \(p_i\) can be reconstructed exactly using the same equations.

The time domain describes how the signal evolves over time, whereas the frequency domain expresses
the same information in terms of the amplitude and phase distribution across frequencies.


\section{The loudspeaker impulse response and its Fourier transform}

Imagine a test signal that contains all frequencies at once, each with the same amplitude and zero phase; that is, $A_j = \text{const.}$ and $\varphi_j = 0$ for all $j$ in the frequency domain.  
When such a signal is converted from the frequency domain to the time domain, the result is a very short pulse at time $t_1$ and zero at all other times $t_{i = 2\ldots N}$—a \emph{Dirac delta}.  
In other words, an ideal Dirac pulse excites all frequencies simultaneously and with equal amplitude.

If we now drive a loudspeaker with this impulse, the microphone records its acoustic output over time.  
The resulting waveform, $h(t)$, is the impulse response, expressed in sound pressure (Pa) as a function of time.  
Because the input contains all frequencies equally, any variation in the measured response reflects how the loudspeaker modifies the amplitude and phase of those frequencies—revealing its deviation from a perfectly flat, ideal response.

The impulse response thus captures the complete acoustic behaviour of the loudspeaker (and the surrounding room): every resonance, reflection, and delay leaves its fingerprint in $h(t)$.  
Early peaks correspond to the direct sound from the driver, later ripples often indicate cabinet or cone resonances, and delayed arrivals show reflections from walls or nearby surfaces.  
Taken together, these features form the bridge between the time and frequency domains.


[INSERT FIGURE HERE -- PANEL (A): DIRAC DELTA (SPEAKER INPUT) -- PANEL (B): SPEAKER IMPULSE RESPONSE (WITH ANNOTATIONS OF FEATURES AS MENTIONED IN TEXT) *** USE REAL MEASURED DATA]


\section{From impulse response to the SPL curve}

\begin{itemize}[noitemsep]
    \item Applying a Fourier transform to \(h(t)\) yields \(H(f)\), the loudspeaker’s frequency response.
    \item The magnitude of \(H(f)\) indicates the relative sound level of each frequency; the phase describes timing and coherence.
    \item With microphone calibration, magnitudes can be expressed as sound pressure level (SPL) in decibels (re~20~µPa).
    \item The transformation from \(h(t)\) to \(H(f)\) turns a single broadband measurement into the familiar SPL vs.\ frequency curve.
\end{itemize}

\vspace{1em}
\noindent\textbf{Figure 4:} Impulse response and its corresponding SPL frequency response.

\section{The Good, the Bad, and the Ugly (Preview)}

\begin{itemize}[noitemsep]
    \item The following sections discuss how real-world processing shapes this transformation:
    \begin{itemize}
        \item \textbf{The Good:} gating and windowing to suppress reflections.
        \item \textbf{The Bad:} excessive windowing that removes low-frequency content.
        \item \textbf{The Ugly:} manipulation of data through zeroing, smoothing, or padding.
    \end{itemize}
    \item Each step is an attempt to recover the true loudspeaker behaviour from a reflection-rich environment—but not all are equally honest.
\end{itemize}

\vspace{1em}
\noindent\rule{\textwidth}{0.4pt}\\
\textit{This conceptual introduction explains how an impulse response represents a complete description of a loudspeaker’s behaviour. Later sections explore how this data is processed, interpreted, and occasionally abused.}

\end{document}
