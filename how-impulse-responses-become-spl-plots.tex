\documentclass[12pt,a4paper]{article}
\usepackage[margin=2.5cm]{geometry}
\usepackage{helvet}
\renewcommand{\familydefault}{\sfdefault}
\usepackage[helvet]{sfmath}
\usepackage{amsmath,amssymb}
\usepackage{enumitem}
\usepackage{titlesec}
\usepackage{graphicx}
\usepackage[hidelinks]{hyperref}
\usepackage[parfill]{parskip}
\usepackage{siunitx}
\usepackage{graphicx}
\graphicspath{{figures/}}

\providecommand{\seclabel}[1]{\label{sec:#1}}
\providecommand{\secn}[1]{Sec.~\ref{sec:#1}}
\providecommand{\figlabel}[1]{\label{fig:#1}}
\providecommand{\figr}[1]{Fig.~\ref{fig:#1}}
\providecommand{\pRef}{p_\mathrm{ref}}

\title{How Loudspeaker Impulse Responses Become Sound~Pressure~Level~(SPL) Plots:\\[0.7em]\textit{The Good, the Bad, and the Ugly}}
\author{Matthias S.~Brennwald}
\date{Document version \today}

\begin{document}
\maketitle

\section{Introduction}

Every loudspeaker designer or tinkerer wants to know how well their loudspeaker performs acoustically. Does it reproduce all frequencies evenly, or does it emphasise some and hide others? The answer lies in its SPL frequency response curve, which shows how loud the speaker is at each frequency.

Understanding and visualising the SPL response is central to almost every aspect of loudspeaker design and evaluation—from developing new drivers to characterising enclosures and interpreting room interactions. The goal seems simple: determine how a loudspeaker responds to different frequencies. The challenge lies in doing so accurately and efficiently, without (unintended) cheating.

An intuitive approach to determine the SPL curve of a loudspeaker would be to play a sine wave with a given frequency at a time, measure the resulting sound pressure level (SPL), repeat the process at the next frequency, and plot the SPL results across frequency. However, this method is very time-consuming and works reliably only in an anechoic environment, as room reflections and echoes can severely distort the measurement. Modern techniques achieve the same result in a single measurement step: they excite the loudspeaker with a short, broadband signal and use the Fourier transform to reveal the frequency response from the measured time-domain data.

This document introduces the concepts that link time-domain measurements (such as impulse responses) to frequency-domain data (such as SPL curves). It aims to provide a practical, intuitive explanation of what the Fourier transform does for loudspeaker measurements, and why understanding the technique's limitations is just as important as applying it. Later sections explore both the strengths and the pitfalls of this process: the good, the bad, and occasionally the ugly.


\section{Time Domain vs.\ Frequency Domain: The Fourier Transform}\seclabel{Fourier_Theory}

Consider a sound pressure signal evolving as a function of time, $x(t)$. The signal is recorded by a microphone and digitized by an A/D converter with sampling rate \(F_s\). As a result, we obtain $N$ discrete samples $x_i = x(t_i)$ at times
\[
t_i = (i-1) \times \tau, \qquad \tau = \frac{1}{F_s}, \quad i = 1, 2, \ldots, N.
\]

The idea of the Fourier transform is to describe each time-domain data point \(x_i\)
as a combination of sine waves, each with amplitude \(A_j\) and phase \(\varphi_j\),
at discrete frequencies
\[
f_j = \frac{F_s}{N}j, \quad j = 1, 2, \ldots, M,
\]
where \(M = (N - 1)/2\) if \(N\) is odd, or \(M = N/2 - 1\) if \(N\) is even.

The discrete Fourier expansion is then written as a system of $N$ equations ($i = 1\ldots N$):
\begin{align}
\text{For odd } N:\quad
x_i &= A_0 + \sum_{j=1}^{M} A_j \cos\!\big(2\pi f_j t_i + \varphi_j\big), \\[6pt]
\text{For even } N:\quad
x_i &= A_0 + \sum_{j=1}^{M} A_j \cos\!\big(2\pi f_j t_i + \varphi_j\big)
      + A_{N/2}\cos\!\big(\pi\, t_i / \tau\big).
\end{align}

Each amplitude–phase pair \((A_j, \varphi_j)\) represents one Fourier component at frequency \(f_j = jF_s/N\). The $A_0$ coefficient is the DC component of the $x_i$ data.

Note that the total number of coefficients $A_j$ (including $A_0$) and $\varphi_j$ together equals \(N\), so the number of data points in the frequency domain is the same as in the time domain. The above equations therefore represent a fully determined system. The coefficients \((A_j, \varphi_j)\) can be uniquely obtained from the measured samples \(x_i\) by solving this system, typically using efficient numerical algorithms such as the \emph{Fast Fourier Transform (FFT)}. Also, the Fourier transform is fully reversible. Given the frequency-domain data \((A_j, \varphi_j)\), the corresponding time-domain signal \(x_i\) can be reconstructed exactly using the above equations.  

Overall, the time-domain data describe how the signal evolves over time, while the frequency-domain data represent the same signal in terms of its amplitude and phase distribution across frequencies. Both domains contain precisely the same information; they just provide different representations of the very same signal.

\subsection{Periodic Signals!}\seclabel{Fourier_periodic}
An important (but often overlooked) aspect of the discrete Fourier transform is that it implicitly assumes the signal to be periodic with period $T = N \times \tau$.  
This follows directly from the use of cosine function in the above Fourier expansion, which is inherently periodic. The representation of the discrete time-domain data $x_i$ as a sum of cosines implies that this signal repeats indefinitely for $t < 0$ and $t > T$.  
However, in many practical cases the measured signal is not actually periodic (e.g., the impulse response discussed in \secn{impulse_response}). This assumption is not inherently a problem, but it does have consequences: if the start and end points of the time-domain data are not at the same amplitude, the implied periodic continuation introduces an artificial discontinuity at the boundary between repetitions. In the frequency domain, this discontinuity appears as additional spectral content (spectral leakage).

\subsection{Where Did the Other Frequencies Go?}
In practice, most real-world signals are not composed of a finite set of perfectly aligned frequencies $f_j$.  
Instead, their spectral energy almost always lies between the discrete frequencies sampled by the Fourier transform.

In the discrete Fourier transform, the available frequencies form equally spaced bins centered at the $f_j$, each with width $\Delta f = F_s/N = F_s \times \tau$.
A signal whose frequency does not coincide with one of these discrete $f_j$ values will have its energy distributed over multiple bins, a phenomenon known as spectral leakage. While the total energy of the signal remains conserved, leakage can cause sharp spectral features to appear broadened or blurred across multiple frequency bins.

Sampling also imposes two important frequency limits.  
The lowest frequency present in the discrete Fourier transform ($f_1 = F_s/N$) is determined by the total length of the digitized signal, $T = N \times \tau$.  
Because the bin width $\Delta f$ equals $f_1$, the signal duration also defines the frequency resolution of the time-domain data.  
The highest resolvable frequency ($f_M = F_s \times M / N$) is determined by the sampling rate: according to the Nyquist theorem, no frequency above $F_s/2$ can be represented in the digitized data.  
Any analog signal components above this limit will be folded back (or ``aliased'') to appear as spurious lower-frequency components in the sampled data.


\section{The Impulse Response and Its Fourier Transform}

...

\section{From a Real-World Impulse Response to Its SPL Curve: Time Gating (Reality Bites!)}\seclabel{from_IR_to_SPL}

...

\section{Fighting With Reality: Working Around the Rules}

\subsection{Is the Gated Impulse Response Periodic?}\seclabel{windowing}

...

\subsection{Is the Gated Impulse Response Too Short?}\seclabel{zeropadding}

...

\section{Putting the Pieces Together: The Good, the Bad, and the Ugly}\seclabel{putting_pieces_together}

...
\end{document}
