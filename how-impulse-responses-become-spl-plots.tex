\documentclass[12pt,a4paper]{article}
\usepackage[margin=2.5cm]{geometry}
\usepackage{helvet}
\renewcommand{\familydefault}{\sfdefault}
\usepackage[helvet]{sfmath}
\usepackage{amsmath,amssymb}
\usepackage{enumitem}
\usepackage{titlesec}
\usepackage{graphicx}
\usepackage{hyperref}

\title{How Impulse Responses Become Sound~Pressure~Level~(SPL) Plots:\\[0.7em]\textit{The Good, the Bad, and the Ugly}}
\author{Matthias S.~Brennwald}
\date{Document version \today}

\begin{document}
\maketitle

\section{Motivation: Why Frequency Response Matters}

\begin{itemize}[noitemsep]
    \item The goal is to determine how loud a loudspeaker is at each frequency.
    \item A straightforward approach is to play pure tones at different frequencies, record the sound pressure level (SPL), and connect the dots to form a curve of SPL vs.\ frequency.
    \item This method is slow and affected by reflections and echoes, unless performed in an anechoic chamber.
    \item A more efficient and accessible alternative relies on the \textbf{impulse response} and the \textbf{Fourier transform}.
    \item This text is a practical and occasionally irreverent guide to turning impulse-response measurements from the time domain into meaningful SPL data in the frequency domain. It explores how an impulse response represents a loudspeaker's acoustic behaviour and how the data is processed, interpreted, and occasionally abused.
\end{itemize}

\vspace{1em}
\noindent\textbf{Figure 1:} Measuring tone-by-tone versus using a single broadband test.

\section{Time domain vs.\ frequency domain: the Fourier transform}

Assume sound pressure signal evolving as function of time, \(p(t)\). The signal is recorded by a microphone and digitized by an A/D converter with a sampling rate $F_s$. Result: \(N\) discrete samples \(p_i\) at times $t_i = 0, \tau, 2\tau,\ldots, (N-1)\tau$ ($\tau = 1/F_s$).

The idea of the Fourier transform is to describe time-domain data point $p_i$ ($i = 1,\ldots,N$) as a combination of sine waves, each with an amplitude \(A_k\) and phase \(\varphi_k\) at frequencies $f_k = (k-1)F_s/N$:

\begin{align}
\text{If $N$ is even:}\quad
x_i &= \frac{A_1}{N}
      + \frac{A_{N/2+1}}{N}(-1)^{i-1}
      + \frac{2}{N}\sum_{k=2}^{N/2} A_k \sin\!\big(2\pi f_k t_i + \varphi_k\big) \notag\\[0.8em]
\text{If $N$ is odd:}\quad
x_i &= \frac{A_1}{N}
      + \frac{2}{N}\sum_{k=2}^{(N+1)/2} 
      A_k \sin\!\big(2\pi f_k t_i + \varphi_k\big)
\end{align}




Each amplitude–phase pair \((A_k,\varphi_k)\) represents one Fourier component with frequency $f_k = (k-1)F_s/N\).

NOTE THAT NUMBER OF COEFFICIENTS $A_k$ AND $\varphi_k)$ SUMS UP TO $N$. IN OTHER WORDS, THE NUMBER OF DATA POINTS IN THE FREQUENCY-DOMAIN IS ALWAYS EQUAL TO THE NUMBER OF TIME-DOMAIN DATA POINTS. THE ABOVE EQUATIONS THEREFORE DESCRIBE A FULLY DETERMINED EQUATION SYSTEM. THE ($A_k$ AND $\varphi_k)$) CAN THEREFORE BE DETERMINED FROM THE MEASURED $x_i$ BY SOLVING THE EQUATION SYSTEM, USUALLY EMPLYING SUITABLE COMPUTER ALGORITHMS (FAST FOURIER TRANSFORM, FFT).

NOTE THAT THE TRANSFORM ALSO WORKS BACKWARDS, FROM TIME DOMAIN TO FREQUENCY DOMAIN. GIVEN THE FREQUENCY DOMAIN DATA $(A_k,\varphi_k)$, THE CORRESONDING TIME-DOMAIN SIGNAL CAN BE DETERMINED EXACTLY FROM THE ABOVE EQUATIONS.

The time domain describes how the signal evolves over time, whereas the frequency domain expresses the same information in terms of the \emph{amplitude and phase distribution} across frequencies.


\section{How Loud Is the Loudspeaker? One Test for All Frequencies}

\begin{itemize}[noitemsep]
    \item Imagine a test signal that contains all frequencies with equal amplitude and zero phase.
    \item Using the inverse Fourier transform, this frequency-domain description can be converted into the time domain.
    \item The result is a very short pulse—mathematically a \textbf{Dirac delta}, or in practice a broadband \textbf{impulse}.
    \item When such a signal drives the loudspeaker, the microphone records its acoustic output over time.
    \item The resulting waveform, \(h(t)\), is the \textbf{impulse response}, expressed in sound pressure (Pa) as a function of time.
    \item Each peak, dip, and reflection in \(h(t)\) corresponds to how the loudspeaker and room respond to different frequencies.
\end{itemize}

\vspace{1em}
\noindent\textbf{Figure 3:} Ideal impulse (test signal) and measured impulse response of a loudspeaker.

\section{From Impulse to Frequency --- The FFT}

\begin{itemize}[noitemsep]
    \item Applying a Fourier transform to \(h(t)\) yields \(H(f)\), the loudspeaker’s frequency response.
    \item The magnitude of \(H(f)\) indicates the relative sound level of each frequency; the phase describes timing and coherence.
    \item With microphone calibration, magnitudes can be expressed as sound pressure level (SPL) in decibels (re~20~µPa).
    \item The transformation from \(h(t)\) to \(H(f)\) turns a single broadband measurement into the familiar SPL vs.\ frequency curve.
\end{itemize}

\vspace{1em}
\noindent\textbf{Figure 4:} Impulse response and its corresponding SPL frequency response.

\section{The Good, the Bad, and the Ugly (Preview)}

\begin{itemize}[noitemsep]
    \item The following sections discuss how real-world processing shapes this transformation:
    \begin{itemize}
        \item \textbf{The Good:} gating and windowing to suppress reflections.
        \item \textbf{The Bad:} excessive windowing that removes low-frequency content.
        \item \textbf{The Ugly:} manipulation of data through zeroing, smoothing, or padding.
    \end{itemize}
    \item Each step is an attempt to recover the true loudspeaker behaviour from a reflection-rich environment—but not all are equally honest.
\end{itemize}

\vspace{1em}
\noindent\rule{\textwidth}{0.4pt}\\
\textit{This conceptual introduction explains how an impulse response represents a complete description of a loudspeaker’s behaviour. Later sections explore how this data is processed, interpreted, and occasionally abused.}

\end{document}
