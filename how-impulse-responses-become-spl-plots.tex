\documentclass[12pt,a4paper]{article}
\usepackage[margin=2.5cm]{geometry}
\usepackage{helvet}
\renewcommand{\familydefault}{\sfdefault}
\usepackage[helvet]{sfmath}
\usepackage{amsmath,amssymb}
\usepackage{enumitem}
\usepackage{titlesec}
\usepackage[hidelinks]{hyperref}
\usepackage[parfill]{parskip}
\usepackage{siunitx}
\usepackage{graphicx}
\graphicspath{{figures/}}

\providecommand{\seclabel}[1]{\label{sec:#1}}
\providecommand{\secn}[1]{Sec.~\ref{sec:#1}}
\providecommand{\figlabel}[1]{\label{fig:#1}}
\providecommand{\figr}[1]{Fig.~\ref{fig:#1}}
\providecommand{\pRef}{p_\mathrm{ref}}

\title{How Loudspeaker Impulse Responses Become Sound~Pressure~Level~(SPL) Plots:\\[0.7em]\textit{The Good, the Bad, and the Ugly}}
\author{mbrennwa@diyAudio.com}
\date{Document version \today}

\begin{document}
\maketitle

\section{Introduction}

Every loudspeaker designer or tinkerer wants to know how well their loudspeaker performs acoustically. Does it reproduce all frequencies evenly, or does it emphasise some and hide others? The answer lies in its SPL frequency response curve, which shows how loud the speaker is at each frequency.

Understanding and visualising the SPL response is central to almost every aspect of loudspeaker design and evaluation, from developing new drivers to characterising enclosures and interpreting room interactions. The goal seems simple: determine how a loudspeaker responds to different frequencies. The challenge lies in doing so accurately and efficiently, without (unintended) cheating.

An intuitive approach to determine the SPL curve of a loudspeaker would be to play a sine wave with a given frequency at a time, measure the resulting sound pressure level (SPL), repeat the process at the next frequency, and plot the SPL results across frequency. However, this method is very time-consuming and works reliably only in an anechoic environment, as room reflections and echoes can severely distort the measurement. Modern techniques achieve the same result in a single measurement step: they excite the loudspeaker with a short, broadband signal and use the Fourier transform to reveal the frequency response from the measured time-domain data.

This document introduces the concepts that link time-domain measurements (such as impulse responses) to frequency-domain data (such as SPL curves). It aims to provide a practical, intuitive explanation of what the Fourier transform does for loudspeaker measurements, and why understanding the technique's limitations is just as important as applying it. Later sections explore both the strengths and the pitfalls of this process: the good, the bad, and occasionally the ugly.


\section{Time Domain vs.\ Frequency Domain: the Fourier Transform}\seclabel{Fourier_Theory}

Consider a sound pressure signal evolving as a function of time, $x(t)$. The signal is recorded by a microphone and digitised by an analogue-to-digital converter with sampling rate \(F_s\). As a result, we obtain $N$ discrete samples $x_i = x(t_i)$ at times
\[
t_i = (i-1) \times \tau, \qquad \tau = \frac{1}{F_s}, \quad i = 1, 2, \ldots, N.
\]

The idea of the Fourier transform is to describe each time-domain data point \(x_i\)
as a combination of sine waves, each with amplitude \(A_j\) and phase \(\varphi_j\),
at discrete frequencies
\[
f_j = \frac{F_s}{N}j, \quad j = 1, 2, \ldots, M,
\]
where \(M = (N - 1)/2\) if \(N\) is odd, or \(M = N/2 - 1\) if \(N\) is even.

The discrete Fourier expansion is then written as a system of $N$ equations ($i = 1\ldots N$):
\begin{align}
\text{For odd } N:\quad
x_i &= A_0 + \sum_{j=1}^{M} A_j \cos\!\big(2\pi f_j t_i + \varphi_j\big), \\[6pt]
\text{For even } N:\quad
x_i &= A_0 + \sum_{j=1}^{M} A_j \cos\!\big(2\pi f_j t_i + \varphi_j\big)
      + A_{N/2}\cos\!\big(\pi\, t_i / \tau\big).
\end{align}

Each amplitude–phase pair \((A_j, \varphi_j)\) represents one Fourier component at frequency \(f_j = jF_s/N\). The $A_0$ coefficient is the DC component of the $x_i$ data.

Note that the total number of coefficients $A_j$ (including $A_0$) and $\varphi_j$ together equals \(N\), so the number of data points in the frequency domain is the same as in the time domain. The above equations therefore represent a fully determined system. The coefficients \((A_j, \varphi_j)\) can be uniquely obtained from the measured samples \(x_i\) by solving this system, typically using efficient numerical algorithms such as the \emph{Fast Fourier Transform (FFT)}. Also, the Fourier transform is fully reversible. Given the frequency-domain data \((A_j, \varphi_j)\), the corresponding time-domain signal \(x_i\) can be reconstructed exactly from the above equations.  

Overall, the time-domain data describe how the signal evolves over time, while the frequency-domain data represent the same signal in terms of its amplitude and phase distribution across frequencies. Both domains contain precisely the same information; they just provide different representations of the very same signal.

\subsection{Periodic Signals!}\seclabel{Fourier_periodic}
An important (but often overlooked) aspect of the discrete Fourier transform is that it implicitly assumes the signal to be periodic with period $T = N \times \tau$.  
This follows directly from the use of cosine functions in the above Fourier expansion, which are inherently periodic. The representation of the discrete time-domain data $x_i$ as a sum of cosines implies that this signal repeats indefinitely for $t < 0$ and $t > T$.  
However, in many practical cases the measured signal is not actually periodic (e.g., the impulse response discussed in \secn{impulse_response}). This assumption is not inherently a problem, but it does have consequences: if the start and end points of the time-domain data are not at the same amplitude, the implied periodic continuation introduces an artificial discontinuity at the boundary between repetitions. In the frequency domain, this discontinuity appears as additional spectral content (spectral leakage).

\subsection{Where Did the Other Frequencies Go?}
In practice, most real-world signals are not composed of a finite set of perfectly aligned frequencies $f_j$.  
Instead, their spectral energy almost always lies between the discrete frequencies sampled by the Fourier transform.

In the discrete Fourier transform, the available frequencies form equally spaced bins centered at the $f_j$, each with width $\Delta f = F_s/N = F_s \times \tau$.
A signal whose frequency does not coincide with one of these discrete $f_j$ values will have its energy distributed over multiple bins, a phenomenon known as spectral leakage. While the total energy of the signal remains conserved, leakage can cause sharp spectral features to appear broadened or blurred across multiple frequency bins.

Sampling also imposes two important frequency limits.  
The lowest frequency present in the discrete Fourier transform ($f_1 = F_s/N$) is determined by the total length of the digitised signal, $T = N \times \tau$.  
Because the bin width $\Delta f$ equals $f_1$, the signal duration also defines the frequency resolution of the time-domain data.  
The highest resolvable frequency ($f_M = F_s \times M / N$) is determined by the sampling rate: according to the Nyquist theorem, no frequency above $F_s/2$ can be represented in the digitised data.  
Any analogue signal components above this limit will be folded back (or ``aliased'') to appear as spurious lower-frequency components in the sampled data.


\section{The Impulse Response and its Fourier Transform}\seclabel{impulse_response}

Consider a test signal that contains all frequencies at once, each with the same amplitude and zero phase: $A_j = 1$ and $\varphi_j = 0$ for all $j$. When such a spectrum is converted from the frequency domain to the time domain using the above Fourier expansion, the result is $x_1 = N$ and $x_{i=2\ldots N} = 0$, the discrete analogue of a Dirac delta. In other words, an ideal delta pulse contains all frequencies simultaneously and with equal amplitude and zero phase.

If we now drive a loudspeaker with this impulse, the microphone records its acoustic output over time (\figr{FIGURE1}).  
The resulting waveform, $h(t)$, is the impulse response, expressed in sound pressure (Pa) as a function of time (\mbox{\figr{FIGURE1}-B}). Because the input contains all frequencies equally, any variation in the measured response reflects how the loudspeaker modifies the amplitude and phase of those frequencies, revealing its deviation from a perfectly flat, ideal response.\footnote{In practice, a very large impulse is required to achieve a good signal-to-noise ratio of the measurement. Most measurement tools therefore do \emph{not} use a Dirac impulse to determine the impulse-response, but rely on maximum-length sequences, broadband noise, or sine sweeps to excite the loudspeaker. The recorded loudspeaker response is then deconvolved from the test signal to obtain the impulse response. The result is the same as in our thought experiment, but with less measurement noise.}

Imagine the measurement of the impulse response were carried out in an anechoic environment. The resulting impulse response would capture the complete acoustic behaviour of the loudspeaker (\mbox{\figr{FIGURE1}-B}):\footnote{These data were measured outdoors in an empty space, with the loudspeaker and the microphone elevated high above the ground to keep away sound reflections from the floor.} every resonance, reflection, and delay leaves its fingerprint in $h(t)$. Early peaks correspond to the direct sound from the driver, and later ripples often indicate cabinet or cone resonances, which at some point die off in the noise floor of the measurement.

In practice, however, measurements usually cannot be carried out in a perfectly anechoic environment. The sound radiated from the speaker gets reflected at the floor, the ceiling and the walls of the room, and furniture or other objects. These reflections show up in the measurement as small wiggles after the main peak of the direct sound from the speaker (\mbox{\figr{FIGURE1}-C}).

\begin{figure}[tbp]
  \begin{center}
    \includegraphics[width=0.7\textwidth]{FIGURE1}
    \caption{Loudspeaker impulse response. (A) The Dirac signal input to the loudspeaker, (B) loudspeaker impulse response in an anechoic environment (which is usually not accessible with a typical measurement setup), (C) impulse response of the same loudspeaker, but from a more typical measurement setup resulting in echoes from the room walls.}
    \figlabel{FIGURE1}
  \end{center}
\end{figure}


\section{From a Real-World Impulse Response to its SPL Curve: Time Gating (Reality Bites!)}\seclabel{from_IR_to_SPL}

We have seen how a single measurement of the impulse response contains everything needed to determine a loudspeaker’s acoustic behaviour.
But, before applying the Fourier transform to the time-domain measurement, let's take a closer look at \mbox{\figr{FIGURE1}-C}:
\begin{itemize}
\item The beginning of the impulse-response measurement (green, 0--\SI{3.08}{ms}) represents the flight time of the sound from the loudspeaker to the microphone and contains no useful information about the loudspeaker itself.
\item The initial part of the impulse response (blue, 3.08--\SI{6.23}{ms}) shows the clean, direct sound from the loudspeaker.
\item After the arrival of the first echo at \SI{6.23}{ms}, the direct signal becomes masked by reflections from the room.
This echoic part of the signal (red) still contains the loudspeaker’s output, but it can no longer be used to assess the loudspeaker’s own behaviour.
\end{itemize}

Only the anechoic segment of the impulse response can therefore be used to characterise the loudspeaker’s behaviour under anechoic conditions. The flight-time and echoic parts of the impulse-response measurement are discarded. In other words, the impulse response is gated to its anechoic part. The Fourier transform of this gated impulse response then yields the amplitude coefficients $A_j$, and hence the loudspeaker’s clean, anechoic SPL data in units of dB-SPL is:
\[
\mathrm{SPL}(f_j) = 20 \log_{10}\left( \frac{A_j / \sqrt{2}}{\pRef} \right),
\]
where $\pRef = \SI{20}{\micro\pascal}$ is the SPL reference pressure.

\figr{FIGURE2} shows the SPL data determined from the anechoic part of the impulse response shown in \figr{FIGURE1}. The number of time-domain data points in the anechoic segment is $N = 140$. The frequency-domain data contains the same number of data points, split between amplitude ($A_j$) and phase ($\varphi_j$) coefficients (see \secn{Fourier_Theory}). The anechoic time-domain segment is $T = \SI{3.15}{ms}$ long, so the lowest frequency ($f_1$) and the width of the frequency bins ($\Delta f$) are $f_1 = \Delta f = 1/T \approx \SI{317}{Hz}$. The ``rough'' presentation of the SPL data in \figr{FIGURE2} may seem awkward, but it clearly exposes the limits of the method with respect to the low-frequency extension and frequency resolution (bin width) of the SPL data.

\begin{figure}[tbp]
  \begin{center}
    \includegraphics[width=0.7\textwidth]{FIGURE2}
    \caption{SPL data determined from the anechoic part of the impulse response in \figr{FIGURE1} (see text). The dots show the frequency-domain data ($\mathrm{SPL}(f_j)$ and $\varphi_j$). The stair steps indicate the frequency bin width ($\Delta f$), which is constant across the entire plot but appears distorted due to the logarithmic scaling of the frequency axis.}
    \figlabel{FIGURE2}
  \end{center}
\end{figure}

The following sections discuss some commonly used techniques to tweak the anechoic data or to make it look prettier. But always remember: \emph{looks can be deceiving!}


\section{Fighting with Reality: Working Around the Rules}

The relationships described in \secn{Fourier_Theory} define the fundamental, mathematical rules on which the Fourier transform operates. However, real-world measurements have a tendency to not play nice with these rules, which is why techniques to reduce frequency leakage or to improve the visual appearance of an SPL curve are applied; sometimes for the better, sometimes not.


\subsection{Is the Gated Impulse Response Periodic?}\seclabel{windowing}

\figr{FIGURE3} shows the implied periodic signal underlying the Fourier transform (see \secn{Fourier_periodic}). As discussed in \secn{Fourier_Theory}, the discrete Fourier transform implicitly assumes that the time-domain signal repeats itself endlessly. In reality, however, a gated impulse response ends abruptly, and its start and end points rarely match. When the signal is treated as periodic, this mismatch produces a small step between the end of one cycle and the beginning of the next.

\begin{figure}[tbp]
  \begin{center}
    \includegraphics[width=\textwidth]{FIGURE3}
    \caption{Periodic expansion (grey) of the gated impulse response (black) as implied by the discrete Fourier transform.}
    \figlabel{FIGURE3}
  \end{center}
\end{figure}

This sharp discontinuity has consequences in the frequency domain. The abrupt jump acts like a sudden transition in the waveform, which the Fourier transform represents as additional high-frequency content. These components do not pertain to the loudspeaker; they are artefacts of how the truncated signal is interpreted, not of the measurement itself.

A common way to reduce this effect is to apply a window function\footnote{There is a whole zoo of window functions, but their individual characteristics are beyond the scope of this document.} to the gated impulse response before taking the Fourier transform (\figr{FIGURE4}). The idea is simple: smooth the end of the signal so that it gently approaches the same value as the beginning, which is typically close to zero within the background noise. This removes or softens the artificial step that would otherwise appear at the boundary of the periodic signal.

\begin{figure}[tbp]
  \begin{center}
    \includegraphics[width=0.7\textwidth]{FIGURE4}
    \caption{Multiplying the gated impulse response (grey) by a window function (red) to obtain a quasi-periodic impulse response (black).}
    \figlabel{FIGURE4}
  \end{center}
\end{figure}

Although windowing reduces unwanted high-frequency artefacts and yields a cleaner-looking SPL curve, it inevitably modifies the data. The window attenuates the tail of the impulse response and therefore weakens, or even removes, some of the valid information contained there (\figr{FIGURE4}). The result may be visually smoother but is no longer a perfectly faithful representation of the original measurement data.

In summary, windowing helps the impulse response conform to the assumptions of the Fourier transform and suppresses artefacts caused by the artificial periodic continuation. However, it must be used with care: it clarifies the true behaviour of the loudspeaker only as long as it does not erase the very details that describe it.


\subsection{Is the Gated Impulse Response too Short?}\seclabel{zeropadding}

The example in \secn{from_IR_to_SPL} has shown how the delay time of the first echo relative to the direct sound from the speaker limits both the low-frequency extension and the frequency resolution of the anechoic response. These two quantities are linked: both depend on the total duration of the usable time-domain signal. A longer signal corresponds to a smaller lowest resolvable frequency $f_1 = 1/T$ and a finer frequency-bin spacing $\Delta f = 1/T$ in the Fourier domain.

To extract more low-frequency information from the measurement, one might be tempted to extend the gate time beyond the arrival of the first few echoes. However, this inevitably contaminates the loudspeaker signal with room reflections and corrupts the anechoic SPL curve, and is therefore not a viable solution.

Another technique often used to make the anechoic data segment appear longer is to artificially pad it with zeros. This produces a longer time-domain signal and therefore a lower apparent low-frequency limit and finer bin spacing in the frequency domain. However, zero-padding does not magically recover the missing low-frequency information; it merely adds more cosine terms to the Fourier expansion. This mathematical extension performs a trigonometric interpolation between the existing frequency-domain data points, as implied by the Fourier series representation. The longer, zero-padded signal therefore yields an SPL curve with more points and an apparent extension toward lower frequencies, along with a smoother visual appearance. Nevertheless, the true low-frequency limit, the effective frequency resolution, and the information content remain unchanged. The zeros have not added anything that was not already present in the measured data.

Always remember: there is no way to extract reliable information on the anechoic response below $f_1$.


\section{Putting the Pieces Together: The Good, the Bad, and the Ugly}\seclabel{putting_pieces_together}

The computation of the anechoic SPL response of a loudspeaker involves a number of processing steps. Each step aims to recover the true behaviour of the loudspeaker from a measurement that is inevitably affected by room echoes and noise. Depending on how they are applied, some steps help to clarify the data, while others may risk distorting or disguising it.

To illustrate this, let us revisit the SPL curve from \figr{FIGURE2} and examine what happens when the data are processed to make the curve appear smoother and visually more appealing. Specifically, we (i) apply a window function as shown in \figr{FIGURE4} and (ii) pad the resulting impulse response with \SI{0.1}{s} of zeros. The resulting SPL curve is shown in \figr{FIGURE5}, together with the original gated response (\figr{FIGURE2}) and the anechoic reference derived from the full \SI{15}{ms} impulse response (\mbox{\figr{FIGURE1}-B}), which represents the best available approximation of the “truth.”

\begin{figure}[tbp]
  \begin{center}
    \includegraphics[width=0.7\textwidth]{FIGURE5}
    \caption{SPL curve determined by windowing and zero-padding the gated impulse response (\secn{putting_pieces_together}, black curve; the dashed part indicates the region not supported by data), in comparison with the gated response (same as in \figr{FIGURE2}, blue) and the anechoic reference (\mbox{\figr{FIGURE1}-B}, red).}
    \figlabel{FIGURE5}
  \end{center}
\end{figure}

The processed SPL curve now extends below \SI{10}{Hz}. However, the dashed portion of the curve below $f_1 = \SI{315}{Hz}$ is a by-product of the zero-padding. These SPL values are not based on measured data and are clearly too low compared to the anechoic reference measurement. At frequencies above $f_1$, the processed SPL curve follows the anechoic reference fairly well, although it misses some of the finer details up to about \SI{3}{kHz}. This loss of detail results from the window function, which attenuates information contained in the tail of the gated impulse response. At higher frequencies, which are dominated by the early part of the impulse response, this effect becomes negligible.

The process of deriving an anechoic SPL response from an impulse-response measurement is always a compromise between mathematical refinement and physical realism. Some operations reveal the underlying loudspeaker behaviour, while others unintentionally obscure it. No matter how attractive the final SPL curve may appear, it must always be accompanied by a clear description of the good and bad things you did to the data; otherwise, the pretty curve decays gracefully into nothing more than an ugly deception.


\end{document}
